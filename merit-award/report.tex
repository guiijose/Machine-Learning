\documentclass[a4paper,12pt]{article}

% Packages
\usepackage[utf8]{inputenc}
\usepackage{amsmath, amssymb}
\usepackage{graphicx}
\usepackage{hyperref}
\usepackage{geometry}
\usepackage{booktabs}
\usepackage{float}
\geometry{margin=1in}

% Title Page
\title{Merit Prize Challenge 2024/2025 \\ \vspace{0.5cm} \large Breast Cancer Dataset Analysis}
\author{Guilherme José & José Caldeira}
\date{November 2024}

\begin{document}

\maketitle
\tableofcontents
\newpage

% Sections
\section{Introduction}
\label{sec:introduction}
Overview of the challenge, dataset, objectives, and the methodology followed.

\subsection{Overview}
In this task we aim to develop a model that can accurately classify a patient with or without cancer based on available medical data. This is often used to support healthcare professionals in order to enhance efficiency and enabling doctors to help a larger number of patients effectively.
\subsection{Dataset description}
The dataset consists of \textbf{30 numerical variables} and \textbf{1 binary variable}, the target. Patients are classified as having a \textbf{benign tumor} or a \textbf{malignant tumor}, target variable is \textbf{0} or \textbf{1}, respectively. The dataset has \textbf{no missing values}.
\subsection{Logistic Regression}
\subsection{EM Clustering}
\subsection{RBF Network}

\section{Logistic Regression on the original data}
\label{sec:logistic-regression-original}
Details of the logistic regression implementation, evaluation, and accuracy results.

\section{EM Clustering Analysis}
\label{sec:em-clustering}

\subsection{Clustering with Different \(k\) Values and Silhouette Evaluation}
Description of EM clustering performed with varying numbers of clusters (\(k\)) and analysis of clustering quality using the Silhouette score. Identification of the optimal \(k\).

\subsection{Mapping Data into Clustering Probabilities}
Details on how the data points were mapped into cluster membership probabilities using the EM model.

\subsection{Logistic Regression on Mapped Data}
Implementation of logistic regression on the data transformed into clustering probabilities and evaluation of its accuracy. Analysis of the relationship between \(k\), clustering quality, and model performance.


\section{RBF Network Training}
\label{sec:rbf-network}
Training of the RBF network using the clustering with optimal \(k\).

\section{Discussion}
\label{sec:discussion}
Key findings, correlations, and overall insights from the analysis.

\section{Conclusion}
\label{sec:conclusion}
Summary of the challenge, results, and future directions.

% References Section
\section*{References}
\label{sec:references}

\end{document}